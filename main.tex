\documentclass[a4paper,11pt]{report}
\usepackage[utf8]{inputenc}
\usepackage[T1]{fontenc}
\usepackage{lmodern}
\usepackage[ngerman]{babel}
\usepackage[margin=20mm, left=20mm, right=10mm, headheight=15pt, includeheadfoot]{geometry}
\usepackage{fancyhdr}
\usepackage{opensans}
\usepackage{titlesec}
\usepackage{tocloft}
\usepackage{titling}
\usepackage{hyperref}
\usepackage{graphicx}
\usepackage{enumerate}
\usepackage{float}
\usepackage{caption}
\usepackage{listings}
\usepackage{minted}

\usemintedstyle{vs}

% Set default font to OpenSans
\renewcommand*\familydefault{\sfdefault}

% Page style for cover page
\fancypagestyle{cover}{
    \fancyhf{}
    \renewcommand{\headrulewidth}{0pt}
    \renewcommand{\footrulewidth}{0pt}
}

% Page style for main content
\fancypagestyle{main}{
    \fancyhf{}
    \fancyhead[L]{\theauthor}
    \fancyhead[R]{\subject}
    \fancyfoot[C]{\thepage}
    \renewcommand{\headrulewidth}{0.4pt}
    \renewcommand{\footrulewidth}{0pt}
}

\titleformat{\chapter}[block]
  {\normalfont\Large\bfseries} % change \Large to \large or any other size that fits
  {\thechapter}
  {1em}
  {}

  \titlespacing*{\chapter}{0pt}{*4}{*2.5}

% Cover page
\newcommand{\coverpage}{
    \thispagestyle{cover}
    \begin{center}
        {\includegraphics[height=3cm]{fh-logo}}\\[1cm]
        {\LARGE \thetitle}\\[0.5cm]
        {\large \theauthor}\\
        \href{mailto:97hilfel@gmail.com}{97hilfel@gmail.com}\\
    \end{center}
    \tableofcontents
    \clearpage
}

% Author and subject
\author{Felix Hillebrand}
\newcommand{\subject}{SSD4UE}

\graphicspath{ {./images/} }

% Main document
\begin{document}

\title{SSD4~-~UE03}
\coverpage

\pagenumbering{roman}
\clearpage
\pagenumbering{arabic}
\pagestyle{main}

\chapter{XQuery}
    Übersetzen Sie die nachfolgenden textuellen Angaben in entsprechende XQuery-Ausdrücke,
    sodass die bei jeder Aufgabenstellung angegebene Ausgabe erzeugt wird. 
    Testen Sie diese Ausdrücke auf Basis des bereitgestellten XML-Dokuments \texttt{ue3\_1\_fitnessdokument.xml}. 
    Die Verwendung von XQuery 1.0 ist ausreichend. 
    Der betroffene Namensraum aus dem XML-Dokument soll jeweils beibehalten werden.

    \begin{enumerate}[1.]
    % 1.1
    \item Selektieren Sie die Messwert-Elemente,
    die zu Gewicht-Messungen gehören und sortieren Sie aufsteigend nach dem Datum der Messung.
    Umgeben Sie das gesamte Resultat mit einem Element \texttt{<M>}.
    \textit{(1,5 Punkt)}

    \begin{minted}[frame=lines, breaklines, linenos]{xml}
<M xmlns="http://www.fitness.at/fitness">
    <Messwert typ="Gewicht" wert="80.7" einheit="kg"/>
    <Messwert typ="Gewicht" wert="80.6" einheit="kg"/>
</M>
    \end{minted}
    \textbf{Lösung:}
    \inputminted[frame=lines, breaklines, linenos]{xquery}{assets/ue3_1_1.xq}
    \newpage

    % 1.2
    \item Selektieren Sie Messung-Elemente mit Blutdruck-Werten und verarbeiten Sie nur
    Messungen, die im Kommentar (Notiz) 'maschinell' (case-insensitiv) enthalten. Erstellen
    Sie ein neues Blutdruck-Element mit dem Zeitpunkt der Messung „zeit“ als Attribut.
    Fügen Sie zwei Kind-Elemente hinzu, „Sys“ für den systolischen Wert und „Dia“ für den
    diastolischen Wert. Packen Sie die Ausgabe in ein „RR“-Element. Verwenden Sie zur
    Formatierung der Ausgabe Element- und Attributkonstruktoren. \textit{(1,5 Punkte)}
    \begin{minted}[frame=lines, breaklines, linenos]{xml}
<RR xmlns="http://www.fitness.at/fitness">
    <Blutdruck zeit="2024-01-28T07:55:42">
        <Sys>121.0</Sys>
        <Dia>70.0</Dia>
    </Blutdruck>
    <Blutdruck zeit="2024-01-27T07:48:28">
        <Sys>115.0</Sys>
        <Dia>67.0</Dia>
    </Blutdruck>
</RR>
    \end{minted}
    \textbf{Lösung:}
    \inputminted[frame=lines, breaklines, linenos]{xquery}{assets/ue3_1_2.xq}
    \newpage

    % 1.3
    \item Erstellen Sie eine Aktivitätsübersicht für alle Geräte, mit denen bereits mindestens 2
    Messvorgänge durchgeführt wurden, führen Sie dazu auch die Zeitpunkte an. 
    Erstellen Sie hierfür ein Element '\texttt{Aktivität}' und listen Sie darin '\texttt{Gerät}'-Elemente 
    (mit der '\texttt{id}' als Attribut) auf. 
    Die Kindelemente '\texttt{Online}' von '\texttt{Gerät}' enthalten die Messzeitpunkte. 
    \textit{(1,5 Punkte)}
    \begin{minted}[frame=lines, breaklines, linenos]{xml}
<Aktivität xmlns="http://www.fitness.at/fitness">
    <Gerät id="mg_bd1">
        <Online>2024-01-28T07:55:42</Online>
        <Online>2024-01-27T07:48:28</Online>
    </Gerät>
    <Gerät id="mg_wg1">
        <Online>2024-01-28T08:45:29</Online>
        <Online>2024-01-22T09:05:51</Online>
    </Gerät>
    <Gerät id="mg_bz1">
        <Online>2024-01-28T08:44:47</Online>
        <Online>2024-01-27T07:49:13</Online>
    </Gerät>
</Aktivität>
    \end{minted}
    \textbf{Lösung:}
    \inputminted[frame=lines, breaklines, linenos]{xquery}{assets/ue3_1_3.xq}

    % 1.4
    \item Geben Sie die Titel (wenn vorhanden) und Vor- und Nachnamen der Person in einem Element 'Name' aus. 
    Fügen Sie auch das Geburtsdatum als Element 'Geburtsdatum' hinzu und 
    berechnen Sie auch das Alter (Attribut 'alter'). 
    Umgeben Sie die Elemente mit einem Tag 'Person'. \textit{(2 Punkte)}

    Ergänzen und verwenden Sie für die Berechnung des Alters folgenden Code:

    \begin{minted}[frame=lines, breaklines, linenos]{xquery}
let $dob := (:Date of birth / Geburtsdatum einsetzen:)
let $birth := number(concat(substring($dob, 1, 4), substring($dob, 6, 2),
substring($dob, 9, 2)))
let $now := (:now/ jetzt angeben:)
let $current := number(concat(substring($now, 1, 4), substring($now, 6, 2),
substring($now, 9, 2)))
let $age := floor (($current - $birth) div 10000)
    \end{minted}

    \begin{minted}[frame=lines, breaklines, linenos]{xml}
<Person xmlns="http://www.fitness.at/person">
    <Name>Sandra Sportlich</Name>
    <Geburtsdatum alter="52">1971-08-24</Geburtsdatum>
</Person>
<!-- Titel zum Testen in XML-Dokument einfügen -->
<Person xmlns="http://www.fitness.at/person">
    <Name>Mag. Sandra Sportlich MSc</Name>
    <Geburtsdatum alter="52">1971-08-24</Geburtsdatum>
</Person>
    \end{minted}
    \textbf{Lösung:}
    \inputminted[frame=lines, breaklines, linenos]{xquery}{assets/ue3_1_4.xq}
    \newpage

    % 1.5
    \item Erstellen Sie ein XML-Fragment "Vitalwerte", das alle Messungen mit Messwerten listet.
    Ergänzen Sie die Messwerte um ein Attribut „flag“ wenn der Messwert außerhalb angegebener Grenzwerte liegt. 
    Berücksichtigen Sie nur Grenzwerte für den passenden Mess-Typ und beachten Sie auch das Gültigkeitsdatum. 
    Liegt kein zutreffender Grenzwert vor bzw. ist der gemessene Wert innerhalb der Grenzwerte, entfällt das Attribut "flag".
    Das Attribut enthält "L" für Werte unter dem unteren Grenzwert bzw. "H" für Werte über dem oberen Grenzwert. \textit{(2,5 Punkte)}

    \textit{Hinweis: Verwenden Sie xs:number(...) für die Umwandlung der Grenzwerte zum Wertevergleich.}

    \begin{minted}[frame=lines, breaklines, linenos]{xml}
<Vitalwerte xmlns="http://www.fitness.at/fitness">
    <Messung typ="BlutdruckPuls" zeitpunkt="2024-01-28T07:55:42">
        <Messwert typ="Systolisch" wert="121.0" einheit="mmHg"/>
        <Messwert typ="Diastolisch" wert="70.0" einheit="mmHg"/>
        <Messwert typ="Puls" wert="79.0" einheit="Schläge/min"/>
    </Messung>
    <Messung typ="Gewicht" zeitpunkt="2024-01-28T08:45:29">
        <Messwert typ="Gewicht" wert="80.6" einheit="kg"/>
    </Messung>
        <Messung typ="Schritte" zeitpunkt="2024-01-28T00:00:00">
        <Messwert typ="Schritte" wert="9800" einheit=""/>
    </Messung>
    <Messung typ="BlutdruckPuls" zeitpunkt="2024-01-27T07:48:28">
        <Messwert typ="Systolisch" wert="115.0" einheit="mmHg"/>
        <Messwert typ="Diastolisch" wert="67.0" einheit="mmHg"/>
    </Messung>
    <Messung typ="Gewicht" zeitpunkt="2024-01-22T09:05:51">
        <Messwert typ="Gewicht" wert="80.7" einheit="kg"/>
    </Messung>
        <Messung typ="Schritte" zeitpunkt="2024-01-20T00:00:00">
        <Messwert typ="Schritte" wert="6320" einheit="pro Tag" flag="L"/>
    </Messung>
        <Messung typ="Blutzucker" zeitpunkt="2024-01-28T08:44:47">
        <Messwert typ="Blutzucker" wert="155.5" einheit="mg/dl"/>
    </Messung>
        <Messung typ="Blutzucker" zeitpunkt="2024-01-27T07:49:13">
        <Messwert typ="Blutzucker" wert="201.5" einheit="mg/dl" flag="H"/>
    </Messung>
</Vitalwerte>       
    \end{minted}
    \textbf{Lösung:}
    \inputminted[frame=lines, breaklines, linenos]{xquery}{assets/ue3_1_5.xq}
    \newpage

    \end{enumerate}

\chapter{XSLT}

Erstellen Sie ein Stylesheet (fitnessdokument.xslt), das zur Transformation des XML-
Dokuments ue3\_2\_fitnessdokument.xml in ein HTML-Dokument verwendet werden kann.
Ergänzen Sie hierfür die Vorgabe in fitnessdokument.xslt. Sie müssen hierbei keine CSS-
Klassen explizit verwenden. Achten Sie darauf, dass das Stylesheet für verschiedene
Fitnessdokumente funktioniert und keine sinnlosen Ausgaben produziert ~-~ z.B. 0-n Medikamente,
0-n akademische Titel, ...

\textit{Hinweis}: Verwenden Sie XSLT 2.0, um auch auf fortgeschrittene Formatierungsfunktionen Zugriff zu erhalten 
(z.B. \href{https://www.w3.org/TR/xslt20/#format-date}{https://www.w3.org/TR/xslt20/\#format-date} 
für die Anpassung der Zeitstempel; wie z.B. \texttt{format-dateTime(...)} oder \texttt{format-date(datum, "[D].[M].[Y]")})

\begin{enumerate}[a.]

    \item Binden Sie das Dokument \texttt{templates.xslt} in Ihr \texttt{fitnessdokument.xslt} ein, sodass die
    benannten Templates verwendet werden können und setzen Sie diese nach Möglichkeit ein.

    \item Geben Sie zu Beginn die Basisdaten des Fitnessdokuments, sowie die Personendaten aus.
    
    \item Erstellen Sie ein benanntes Template \texttt{printMessungTabelle}. 
    Als Parameter des Templates definieren und verwenden Sie den Messungs-Typ und die Messungen. 
    Verwenden Sie nach Möglichkeit bereits vorgegebene benannte Templates \texttt{printUserFormattedDateTime},
    \texttt{printWertEinheit} und \texttt{getSeriennummerForMessgerätId}. 
    Sortieren Sie die einzelnen Messungen.

    \item Wird der Knoten Vitaldaten selektiert, rufen Sie das neue Template \texttt{printMessungTabelle} auf.

    \item Ergänzen Sie Templates zur Ausgabe der Messgeräte bzw. des jeweiligen Messgeräts bzw. Gerätetyp, Hersteller, Modellnummer, Seriennummer.
    
    \item Ergänzen Sie die jeweiligen Templates, sodass die Medikationsliste bzw. die Medikation entsprechend der Vorgabe ausgegeben wird.
    \item Ergänzen Sie in templates.xslt die Definition für den unteren bzw. oberen Schwellwert ausgehend vom Parameter messung. 
    Sie können davon ausgehen, dass der Parameter messung einer Messung entspricht.
    \item Ergänzen Sie weiters: Liegt der Wert ober bzw. unterhalb der Schwellwerte soll \texttt{++} bzw. \texttt{--} angezeigt werden.
    \item Verwenden Sie das Dokument \texttt{ue3\_2\_Fitnessdokument\_spezial.xml} und prüfen Sie, ob Ihr Stylesheet aus Beispiel 1 auch Sonderfälle ordentlich verarbeiten kann. 
    Passen Sie Ihr \texttt{fitnessdokument.xslt} oder \texttt{templates.xslt} gegebenenfalls diesbezüglich an.
    \item Die Verarbeitung mit XSLT kann sowohl pull- als auch push-basiert erfolgen. 
    Ihr erstelltes Stylesheet verwendet eine Mischung aus diesen beiden Verarbeitungsarten. 
    Kennzeichnen Sie die jeweiligen Templates bzw. Abschnitte in Ihrem Stylesheet mit der Art der Verarbeitung (push-basiert / pull-basiert).
\end{enumerate}


\textbf{Fitnessdokument:}
\inputminted[frame=lines, breaklines, linenos]{xslt}{assets/fitnessdokument.xslt}

\textbf{Templates:}
\inputminted[frame=lines, breaklines, linenos]{xslt}{assets/templates.xslt}

\textbf{Screenshot:}
\begin{center}
    \includegraphics[width=\textwidth, height=\textheight]{assets/XSLOutput.png}
\end{center}

\newpage

\chapter{Legal}
Die Ausarbeitung der Aufgabe wurde durch:
\begin{itemize}
    \item \texttt{OpenAI - GPT-4.5 Turbo}, 
    \item \texttt{OpenAI - GPT-4.5 Vision}, 
    \item \texttt{Anthropic -- Claude 3 Opus}, 
    \item \texttt{Kagi - FastGPT}
\end{itemize}
mit mehreren unterschiedlichen Prompts und Custom Instructions sowie die verwendung des \href{https://xkcd.com/323/}{Ballmer Peak} unterstützt.

\end{document}
